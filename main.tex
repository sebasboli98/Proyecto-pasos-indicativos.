\documentclass{article}
\usepackage[utf8]{inputenc}
\usepackage[spanish]{babel}
\usepackage{listings}
\usepackage{graphicx}
\graphicspath{ {images/} }
\usepackage{cite}

\begin{document}

\begin{titlepage}
    \begin{center}
        \vspace*{1cm}
            
        \Huge
        \textbf{Instrucciones translado punto A a punto B}
            
        \vspace{0.5cm}
        \LARGE
        
            
        \vspace{1.5cm}
            
        \textbf{Sebastian Bolivar Vanegas CC. 1017268527}
            
        \vfill
            
        \vspace{0.8cm}
            
        \Large
        Despartamento de Ingeniería Electrónica y Telecomunicaciones\\
        Universidad de Antioquia\\
        Medellín\\
        Marzo de 2021
            
    \end{center}
\end{titlepage}

\tableofcontents
\newpage
\section{Sección introductoria}\label{intro}
Este listado se basa en unos paso a paso lo más detallado posible para que “X” persona desplazara un par te tarjetas en una posición inicial A; en una situación específica, a otra posición “B” únicamente ayudándose de una mano.

\section{Pasos} \label{contenido}
1.Sujetar la hoja.\\
2. Mover la hoja de modo que no se encuentre encima de las tarjetas.\\ 
3. Agarrar las dos tarjetas de modo que se suspendan de la superficie. \\
4. Con cualquiera de los dedos acomodar la hoja de manera centrada que quede cómoda. \\
5. Alinear las tarjetas de manera vertical que queden como si fueran una sola tarjeta.  \\
6. Lentamente vamos bajando las tarjetas hasta hacer un leve contacto con la hoja. \\
7. Con la yema de los dedos vamos liberando la presión de manera que se vayan desprendiendo. \\
8. Lentamente vamos dejando caer un poco la tarjeta sobre la que se está apoyando. \\






\bibliographystyle{IEEEtran}


\end{document}
